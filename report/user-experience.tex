\subsection{User Experience}
\label{sec:user-experience}
Within this subsection we will elaborate on our design choices in user experience design related to locking and hiding of both data and the app.
We also briefly discuss the philosophy behind the styling.


\subsubsection{Locking and Hiding of Data}
Once the application has been opened, the user can interact with the files managed by DroidStealth.
Destroying an encrypted file is allowed, but opening and exporting/sharing files is not; the user will have to decrypt them first.
This extra step is part of the interaction to make the user aware that the files are then decrypted.

To add to this awareness, a warning is shown to the user, explaining that some files are unlocked and pose an exposure risk.
This warning is shown as a persistent notification in the notification drawer of the Android device.
When the notification is pressed, all unlocked files are immediately locked.
This provides a user friendly way to swiftly encrypt the files so they cannot be found, without the need to reopen the app.
\todo{screenshots of notifications.}

Depending on the file size, encrypting or decrypting a file could take up to several minutes.
During this process, a notification which informs the user of the encryption or decryption process.
Once the user taps on it, the encryption or decryption is canceled by emptying the queue and finishing the current task.

\subsubsection{Launching DroidStealth}
The trivial, default way of launching DroidStealth is through the app drawer of an Android device. 
When launching the application the first time, the user is promted to enter a pin that will be used to access the application.
On following launches, the app will present the user with a numeric keyboard to enter its pin code. \todo{screenshot of keyboard.}
If the pin is entered correctly, the DroidStealth will launch into the secret gallery \todo{screenshot of gallery}.
This pin code is required to open the application trough any launch method.
If the user forgets the pin code, the data in the application will remain encrypted forever.

\textbf{Alternative Launch Methods}
As explained in Section~\ref{sec:approach}, DroidStealth provides alternative launch methods.
The application provides a `Launch Menu' where launch methods can be enabled and disabled \todo{screenshot of launch menu}.
To protect the user from not being able to open the application, at least one launch method must be enabled at all times.

When the user adds a DroidStealth launch widget to the device's home screen, the new widget and all existing widgets are made temporarily visible.
This helps the user to place the new widget, as well as retrieve existing (lost) widget locations.
Touching any of the widgets will make all of them transparent again, and therefore invisible to the human eye.
From that point on, when the user presses the invisible widget 5 times, the app is launched, showing the pin keyboard.

\subsection{Styling}

DroidStealth is themed with a dark color in order to give users the feeling that they are indeed working in secret, and that their data will be safely hidden.
To give DroidStealth a professional ambience, we used a rounded but solid straight font for our titles, and a formal font for all other texts.
Combining this with the use of bright, but slightly softened primary and secondary colors, the user is provided with a balanced and consistent user interface.
\todo{ADD SMALL IMAGE}

The use of green and red is mirrored from the conventional meaning of those colors.
Green will always be used to indicate that something is good or safe, red will do the opposite.
Furthermore, we use the color orange when indicating a process is being executed.
Whenever a file is locking, or unlocking, the status bar of the file will be orange, and a animated twirl is shown to indicate that the file is being processed.
\todo{ADD SMALL IMAGE}

Finally, the gallery uses no padding or margins between the thumbnails of files, as we want to optimally make use of all the screen space.
Thumbnails are thus as big as they can be, and thus will be most efficient in showing the user its contents.
